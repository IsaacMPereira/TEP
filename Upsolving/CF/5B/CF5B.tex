\section{Codeforces Beta Round \#5 -- Problem B: Center Alignment}

\begin{frame}[fragile]{Problema}

Almost every text editor has a built-in function of center text alignment. The developers of the popular in Berland text editor «Textpad» decided to introduce this functionality into the fourth release of the product.

You are to implement the alignment in the shortest possible time. Good luck!

\end{frame}

\begin{frame}[fragile]{Entrada e saída}

\textbf{Input}

The input file consists of one or more lines, each of the lines contains Latin letters, digits and/or spaces. The lines cannot start or end with a space. It is guaranteed that at least one of the lines has positive length. The length of each line and the total amount of the lines do not exceed 1000.

\textbf{Output}

Format the given text, aligning it center. Frame the whole text with characters «*» of the minimum size. If a line cannot be aligned perfectly (for example, the line has even length, while the width of the block is uneven), you should place such lines rounding down the distance to the left or to the right edge and bringing them closer left or right alternatively (you should start with bringing left). Study the sample tests carefully to understand the output format better.

\end{frame}

\begin{frame}[fragile]{Exemplo de entradas e saídas}

\begin{minipage}[t]{0.5\textwidth}
\textbf{Sample Input}
\begin{verbatim}
This  is

Codeforces
Beta
Round
5
\end{verbatim}
\end{minipage}
\begin{minipage}[t]{0.45\textwidth}
\textbf{Sample Output}
\begin{verbatim}
************
* This  is *
*          *
*Codeforces*
*   Beta   *
*  Round   *
*     5    *
************
\end{verbatim}
\end{minipage}
\end{frame}

\begin{frame}[fragile]{Solução com complexidade $O(n)$}

    \begin{itemize}
        \item Primeiramente, é preciso determinar o tamanho da maior linha $M$, o qual irá
            determinar a largura do quadro

        \item Inicialmente, deve ser impressa uma linha com $M + 2$ caracteres \lq *\rq

        \item Para cada linha $L$, é preciso confrontar seu tamanho com $M$

        \item Se for menor, a diferença deve ser dividida igualmente entre os lados esquerdo e
            direito

        \item No caso de uma diferença ímpar, o caractere restante deve ser distribuído
            alternadamente, inicialmente à direita

        \item Uma variável auxiliar \textit{padding} pode ser usada para manter esta
            alternância

        \item Finalmente, deve ser impressa uma nova linha com $M + 2$ caracteres \lq *\rq

   \end{itemize}

\end{frame}

\begin{frame}[fragile]{Solução AC}
    \inputsnippet{cpp}{1}{20}{codes/5B.cpp}
\end{frame}

\begin{frame}[fragile]{Solução AC}
    \inputsnippet{cpp}{22}{41}{codes/5B.cpp}
\end{frame}

\begin{frame}[fragile]{Solução AC}
    \inputsnippet{cpp}{43}{63}{codes/5B.cpp}
\end{frame}
