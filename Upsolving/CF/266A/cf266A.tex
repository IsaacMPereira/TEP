\section{Codeforces Round \#163 -- Problem A: Stones on the Table}

\begin{frame}[fragile]{Problema}

There are $n$ stones on the table in a row, each of them can be red, green or blue. Count the minimum number of stones to take from the table so that any two neighboring stones had different colors. Stones in a row are considered neighboring if there are no other stones between them.

\end{frame}

\begin{frame}[fragile]{Entrada e saída}

\textbf{Input}

The first line contains integer $n$ $(1\leq n\leq 50)$ -- the number of stones on the table.

The next line contains string $s$, which represents the colors of the stones. We'll consider the 
stones in the row numbered from 1 to $n$ from left to right. Then the $i$-th character s equals 
``\texttt{R}", if the $i$-th stone is red, ``\texttt{G}", if it's green and ``\texttt{B}", 
if it's blue.

\textbf{Output}

Print a single integer -- the answer to the problem.

\end{frame}

\begin{frame}[fragile]{Exemplo de entradas e saídas}

\begin{minipage}[t]{0.5\textwidth}
\textbf{Entrada}
\begin{verbatim}
3
RRG

5
RRRRR

4
BRBG
\end{verbatim}
\end{minipage}
\begin{minipage}[t]{0.45\textwidth}
\textbf{Saída}
\begin{verbatim}
1


4


0
\end{verbatim}
\end{minipage}
\end{frame}

\begin{frame}[fragile]{Solução com complexidade $O(n)$}

    \begin{itemize}
        \item O problema consiste em determinar o número de remoções a serem realizadas de modo
            que caracteres vizinhos sejam distintos

        \item Para tal, basta observar todos os caracteres de $s$ em sequência, um por vez,
            e manter o registro do último 
            caractere que foi observado

        \item Este registro pode ser inicializado com um valor sentinela que não pode ocorrer
            na string (por exemplo, o caractere espaço em branco)

        \item Caso o caractere a ser observado é diferente do anterior, basta atualizar o
            anterior com o atual e prosseguir

        \item Caso seja idêntico ao anterior, é necessário removê-lo

        \item Neste caso não é necessário atualizar o valor do anterior

        \item Esta solução tem complexidade $O(n)$, pois visita cada caractere uma única vez

   \end{itemize}

\end{frame}

\begin{frame}[fragile]{Solução AC com complexidade $O(n)$}
    \inputsnippet{cpp}{1}{20}{266A.cpp}
\end{frame}

\begin{frame}[fragile]{Solução AC com complexidade $O(n)$}
    \inputsnippet{cpp}{21}{40}{266A.cpp}
\end{frame}
