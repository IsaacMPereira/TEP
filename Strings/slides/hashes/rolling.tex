\section{{\it Polynomial Rolling Hash}}

\begin{frame}[fragile]{Definição}

    \metroset{block=fill}
    \begin{block}{Polynomial Rolling Hash}
    Seja $S$ uma string de tamanho $n$, cujos elementos são indexados de 0 a $n - 1$. A função 
    \begin{align*}
        h(S) &= \left(\sum_{i=0}^{n - 1} S_ip^i\right)\ \mbox{mod}\ q \\
        &= \left(S_0 + S_1p + S_2p^2 + \ldots + S_{n-1}p^{n-1}\right)\ \mbox{mod}\ q,
    \end{align*}
    onde $p$ e $q$ são dois inteiros positivos, é denominada \textit{polynomial rolling hash}.
    \end{block}

\end{frame}

\begin{frame}[fragile]{Escolha dos parâmetros}

    \begin{itemize}
        \item Em geral, $p$ é um número primo aproximadamente igual ao tamanho do alfabeto
        \pause

        \item Para um alfabeto de 26 letras, uma escolha razoável seria $p = 31$
        \pause

        \item Para maiúsculas e minúsculas pode-se adotar $p = 53$
        \pause

        \item O valor de $q$ deve ser grande, pois a chance de colisão entre duas strings
            sorteadas aleatoriamente é de $1/q$
        \pause

        \item Usar um número primo para $q$ também é uma boa escolha, no sentido de evitar 
            colisões
        \pause


        \item O valor $q = 10^9 + 7$ tem a vantagem de ser fácil de lembrar e digitar, e também
            de permitir a multiplicação sem \textit{overflow} usando variáveis do tipo
            \code{c++}{long long} 
    \end{itemize}

\end{frame}

\begin{frame}[fragile]{Mapeamento de caracteres}

    \begin{itemize}
        \item Na definição da função $h$ o valor $s_i$ corresponde ao mapeamento do caractere
        $S[i]$ da string para um inteiro
        \pause


        \item Em termos formais, dado um alfabeto $\Sigma$ e uma função
        \[
            f : \Sigma \to \mathbb{N},
        \]
        então $s_i = f(S[i])$, onde $S[i]\in\Sigma$ para todo $i = 0, 1, 2, \ldots, N - 1$
        \pause


        \item Um mapeamento possível seria $f(\mbox{\code{cpp}{'a'}}) = 1, f(\mbox{\code{cpp}{'b'}}) = 2, \ldots,
        f(\mbox{\code{cpp}{'z'}}) = 26$
        \pause

        \item Veja que o caractere \code{cpp}{'a'} não é mapeado para zero, e sim para um, para
            evitar que todas as strings compostas por repetições deste caractere tenham o 
            mesmo \textit{hash} $h$
    \end{itemize}

\end{frame}

%\begin{frame}[fragile]{Implementação do {\it rolling hash} em Haskell}
%    \inputsnippet{haskell}{1}{19}{codes/h.hs}
%\end{frame}

\begin{frame}[fragile]{Implementação do {\it rolling hash} em C++}
    \inputsnippet{cpp}{1}{21}{codes/h.cpp}
\end{frame}

\begin{frame}[fragile]{Calculo do {\it hash} das substrings de $S$}

    \begin{itemize}
        \item Dada uma string $S$, a definição de $h$ permite computar o valor de $h(S[i..j])$,
            para qualquer par $i\leq j$ de índices válidos, em $O(1)$, se conhecidos os valores
            de $h$ para todos os prefixos $S[0..i]$ de $S$
        \pause

        
        \item A função $h$ é definida por
        \begin{align*}
        h(S) &= \left(\sum_{i=0}^{n - 1} S_ip^i\right)\ \mbox{mod}\ q
        \end{align*}
        \pause


        \item Deste modo,
        \begin{align*}
        h(S[i..j])p^i &= \left(\sum_{k=i}^{j} S_ip^i\right)\ \mbox{mod}\ q \\
        &= \left(h(S[0..j]) - h(S[0..(i - 1)])\right) \ \mbox{mod}\ q
        \end{align*}
    \end{itemize}

\end{frame}

\begin{frame}[fragile]{Calculo do {\it hash} das substrings de $S$}

    \begin{itemize}
        \item Para obter o valor do \textit{hash} para $S[i..j]$ é necessário multiplicar a expressão acima
            pelo inverso multiplicativo $(p^i)^{-1}$ de $p^i$ módulo $q$
        \pause

        \item Este pode ser obtido pelo Pequeno Teorema de Fermat: se $q$ é primo e 
        $(a, q) = 1$, então
        \[
            a^{q-1}\equiv \Mod{1}{q}
        \]
        \pause


        \item Assim, como $q \geq 2$,
        \[
            a\cdot a^{q-2}\equiv \Mod{1}{q},
        \]
        vale que 
        \[
            a^{-1}\equiv \Mod{a^{q-2}}{q}
        \]
        \pause


        \item Se os inversos de $p^i$ também forem precomputados, juntamente com os 
        \textit{hashes} dos prefixos $S[0..i]$, os valores $h(S[i..j])$ podem ser calculados
        em $O(1)$
    \end{itemize}

\end{frame}

%\begin{frame}[fragile]{Contagem das substrings distintas em Haskell}
%    \inputsnippet{haskell}{1}{17}{codes/subs.hs}
%\end{frame}

%\begin{frame}[fragile]{Contagem das substrings distintas em Haskell}
%    \inputsnippet{haskell}{18}{36}{codes/subs.hs}
%\end{frame}

\begin{frame}[fragile]{Contagem das substrings distintas em C++}
    \inputsnippet{cpp}{1}{20}{codes/subs.cpp}
\end{frame}

\begin{frame}[fragile]{Contagem das substrings distintas em C++}
    \inputsnippet{cpp}{22}{40}{codes/subs.cpp}
\end{frame}

\begin{frame}[fragile]{Contagem das substrings distintas em C++}
    \inputsnippet{cpp}{41}{60}{codes/subs.cpp}
\end{frame}

\begin{frame}[fragile]{Contagem das substrings distintas em C++}
    \inputsnippet{cpp}{62}{81}{codes/subs.cpp}
\end{frame}

\begin{frame}[fragile]{Contagem das substrings distintas em C++}
    \inputsnippet{cpp}{82}{102}{codes/subs.cpp}
\end{frame}

\begin{frame}[fragile]{Redução da probabilidade de colisão}

    \begin{itemize}
        \item Dadas duas strings $S$ e $T$ escolhidas aleatoriamente, a probabilidade de
            colisão entre ambas é de $1/q$
        \pause


        \item Assim, com $q = 10^9 + 7$, se $S$ for comparado com $n = 10^6$ strings distintas,
            a probabilidade de acontecer uma colisão é igual a $n/q = 10^3$
        \pause


        \item Um modo de diminuir esta probabilidade é utilizar o \textit{hash} duas vezes
        \pause


        \item Em termos mais preciso, seja $h_i$ a função de \textit{rolling hash} que utiliza
            os parâmetros $p_i$ e $q_i$
        \pause


        \item O \textit{hash} duplo $h_{ij}$ associa uma string $S$ a um par de inteiros da 
            seguinte maneira:
            \[
                h_{ij}(S) = (h_i(S), h_j(S))
            \]
        \pause


        \item Se $q_i, q_j > 10^9$, a função $h_{ij}$ produz mais de $10^{18}$ pares distintos,
            de forma que a comparação de $S$ com $n = 10^6$ strings distintas passa a ter
            probabilidade de colisão igual a $n/(q_iq_j) = 1/10^{12}$
    \end{itemize}

\end{frame}

%\begin{frame}[fragile]{Implementação do {\it hash} duplo em Haskell}
%    \inputsnippet{haskell}{1}{16}{codes/double_hash.hs}
%\end{frame}

\begin{frame}[fragile]{Implementação do {\it hash} duplo em C++}
    \inputsnippet{cpp}{1}{18}{codes/double_hash.cpp}
\end{frame}

\begin{frame}[fragile]{Implementação do {\it hash} duplo em C++}
    \inputsnippet{cpp}{20}{42}{codes/double_hash.cpp}
\end{frame}
