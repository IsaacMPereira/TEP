\section{Variantes do algoritmo de Rabin-Karp}

\begin{frame}[fragile]{Diminuição da complexidade para o cálculo dos {\it hashes}}

    \begin{itemize}
        \item  Da maneira como foi apresentada, o algoritmo de Rabin-Karp tem complexidade 
            $O(mn)$ no pior caso, o mesmo da busca completa, e com \textit{runtime} maior,
            por conta do cálculo dos \textit{hashes}
        \pause

        \item Uma primeira melhoria que pode ser feita é usar o \textit{rolling hash}, e computar
            $h(S[(i+1)..(i+m)])$ a partir de $h(S[i..(i + m -1)])$ com custo $O(1)$
        \pause

        \item Isto é possível, pois se $h_i(S) = h(S[i..(i + m - 1)]$, então
        \begin{small}
        \begin{align*}
            h_{i + 1}(S)
            &= \left(S_{i+1} + S_{i + 2}p + \ldots + S_{i+m}p^{m - 1}\right)\ \mbox{mod}\ q \\
            &= \left(\frac{S_i + S_{i+1}p + \ldots + S_{i+m-1}p^{n - 1} + S_{i+m}p^m - S_i}{p} \right)\ \mbox{mod}\ q \\
            &= \left(\frac{S_i + S_{i+1}p + \ldots + S_{i+m-1}p^{n - 1} - S_i}{p} + S_{i+m}p^{m -1} \right)\ \mbox{mod}\ q \\
            &= \left(\frac{h_i(S) - S_i}{p} + S_{i+m}p^{m -1}\right)\ \mbox{mod}\ q 
        \end{align*}
        \end{small}
         
    \end{itemize}

\end{frame}

\begin{frame}[fragile]{Diminuição da complexidade para o cálculo dos {\it hashes}}

    \begin{itemize}
        \item Observe que a divisão deve ser feita por meio da multiplicação pelo inverso
            multiplicativo de $p$ módulo $q$
        \pause

        \item Assim,
        \[
            h_{i+1}(S) = \left((h_{i} - S[i])p^{-1} + S_{i+m}p^{m -1}\right)\ \mbox{mod}\ q
        \]
        \pause

        \item Se a constante $k\equiv \Mod{p^{m-1}}{q}$ for precomputada, cada atualização do
            \textit{hash} tem custo $O(1)$
        \pause

        \item O inverso $i = \Mod{p^{-1}}{q}$ também pode ser precomputado, como no caso da constante $k$
        \pause

        \item O pior caso ainda tem complexidade $O(nm)$, mas o caso médio passa a ter
            complexidade $O(n + m)$

    \end{itemize}

\end{frame}

\begin{frame}[fragile]{Pseudocódigo do algoritmo de Rabin-Karp}

    \begin{algorithm}[H]
        \caption{Algoritmo de Rabin-Karp com \textit{Rolling Hash}}
        \begin{algorithmic}[1]
            \Require Duas strings $P$ e $S$ e os parâmetros $p$ e $q$ do \textit{rolling hash} $h$
            \Ensure O número de ocorrências $occ$ de $P$ em $S$

            \Function{RabinKarp}{$P$,$S$}
                \State $m \gets |P|$, $n \gets |S|$, $occ \gets 0$
                \State $h_P \gets h(P)$, $h_S \gets h(S[1..m])$
                \State $k \gets \Mod{p^{m - 1}}{q}$, $i \gets \Mod{p^{-1}}{q}$
                \State
                \For {$i \gets 1 \forto n - m + 1$}
                    \If {$h_S = h_P$}
                        \If {$S[i..(i + m - 1)] = P$}
                            \State $occ \gets occ + 1$
                        \EndIf
                    \EndIf

                    \If {$i \neq n - m + 1$}
                        \State $h_S \gets \Mod{\left(i(h_S - S[i]) + kS[i + m]\right)}{q}$
                    \EndIf
                \EndFor

                \State \Return $occ$
            \EndFunction
        \end{algorithmic}
    \end{algorithm}

\end{frame}

%\begin{frame}[fragile]{Implementação do algoritmo de Rabin-Karp em Haskell}
%    \inputsnippet{haskell}{1}{17}{codes/rk2.hs}
%\end{frame}
%
%\begin{frame}[fragile]{Implementação do algoritmo de Rabin-Karp em Haskell}
%    \inputsnippet{haskell}{19}{33}{codes/rk2.hs}
%\end{frame}

\begin{frame}[fragile]{Implementação do algoritmo de Rabin-Karp em C++}
    \inputsnippet{cpp}{1}{20}{codes/rk2.cpp}
\end{frame}

\begin{frame}[fragile]{Implementação do algoritmo de Rabin-Karp em C++}
    \inputsnippet{cpp}{22}{36}{codes/rk2.cpp}
\end{frame}

\begin{frame}[fragile]{Implementação do algoritmo de Rabin-Karp em C++}
    \inputsnippet{cpp}{38}{55}{codes/rk2.cpp}
\end{frame}


\begin{frame}[fragile]{Rabin-Karp com {\it hashes} perfeitos}

    \begin{itemize}
        \item A complexidade do pior caso não se alterou por conta da comparação direta das
            substrings $S[i..j]$ com o padrão $P$ no caso de colisão
        \pause

        \item Esta comparação pode ser eliminada se a função de hash $h$ for perfeita para 
            o conjunto $\mathcal{P}$ tal que
        \[
            \mathcal{P} = \lbrace P \rbrace \cup \lbrace S[i..j]\ |\ i\in [1,n], j\in [i,n]\rbrace
        \]
        \pause

        \item O uso de \textit{hash} duplo pode auxiliar na obtenção de uma função de
            \textit{hash} perfeita, uma vez que
            \[
                |\mathcal{P}| \leq \dfrac{n(n + 1)}{2} + 1
            \]
        \pause
 
        \item A eliminação desta comparação reduz a complexidade do pior caso para 
            $O(n + m)$
    \end{itemize}

\end{frame}

\begin{frame}[fragile]{Pseudocódigo do algoritmo de Rabin-Karp}

    \begin{algorithm}[H]
        \caption{Algoritmo de Rabin-Karp com \textit{hash} perfeito}
        \begin{algorithmic}[1]
            \Require Duas strings $P$ e $S$ e os parâmetros $p, q$ do \textit{rolling hash} perfeito $h$
            \Ensure O número de ocorrências $occ$ de $P$ em $S$

            \Function{RabinKarp}{$P$,$S$}
                \State $m \gets |P|$, $n \gets |S|$, $occ \gets 0$
                \State $h_P \gets h(P)$, $h_S \gets h(S[1..m])$
                \State $k \gets \Mod{p^{m - 1}}{q}$, $i \gets \Mod{p^{-1}}{q}$
                \State
                \For {$i \gets 1 \forto n - m + 1$}
                    \If {$h_S = h_P$}
                        \State $occ \gets occ + 1$
                    \EndIf

                    \If {$i \neq n - m + 1$}
                        \State $h_S \gets \Mod{\left(i(h_S - S[i]) + kS[i + m]\right)}{q}$
                    \EndIf
                \EndFor

                \State \Return $occ$
            \EndFunction
        \end{algorithmic}
    \end{algorithm}

\end{frame}

%\begin{frame}[fragile]{Implementação do algoritmo de Rabin-Karp em Haskell}
%    \inputsnippet{haskell}{1}{17}{codes/rk3.hs}
%\end{frame}

%\begin{frame}[fragile]{Implementação do algoritmo de Rabin-Karp em Haskell}
%    \inputsnippet{haskell}{19}{32}{codes/rk3.hs}
%\end{frame}

\begin{frame}[fragile]{Implementação do algoritmo de Rabin-Karp em C++}
    \inputsnippet{cpp}{1}{19}{codes/rk3.cpp}
\end{frame}

\begin{frame}[fragile]{Implementação do algoritmo de Rabin-Karp em C++}
    \inputsnippet{cpp}{21}{39}{codes/rk3.cpp}
\end{frame}

\begin{frame}[fragile]{Implementação do algoritmo de Rabin-Karp em C++}
    \inputsnippet{cpp}{41}{59}{codes/rk3.cpp}
\end{frame}

\begin{frame}[fragile]{Implementação do algoritmo de Rabin-Karp em C++}
    \inputsnippet{cpp}{61}{77}{codes/rk3.cpp}
\end{frame}
