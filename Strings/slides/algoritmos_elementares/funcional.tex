\section{Strings e programação funcional}

\begin{frame}[fragile]{Motivação}

    \begin{itemize}
        \item Mapas, filtros e reduções são técnicas de programação funcional que permitem alterar os elementos de um vetor, gerar um novo vetor a partir da seleção de elementos específicos de um vetor dado ou gerar um único objeto ou elemento a partir dos elementos de um vetor
        \pause

        \item Sendo uma string um vetor de caracteres, estas técnicas podem ser adaptadas para o contexto da manipulação de textos e caracteres
        \pause

        \item A vantagem de tal abordagem é a redução do tamanho do código, evitando laços e
            variáveis temporárias explícitas
        \pause

        \item Outro aspecto importante é que o uso de tais conceitos permite simplificar a
            notação e descrição de problemas de strings
    \end{itemize}

\end{frame}

\begin{frame}[fragile]{Mapas}

    \begin{itemize}
        \item Um mapa (ou mapeamento) consiste em uma função $m_f: S_N \to S_N$, 
            onde $S_N$ é o conjunto de todas as strings de tamanho $N$ e $f: \Sigma \to \Sigma$ é uma função 
            cujo domínio é o alfabeto $\Sigma$ tal que se $y = m_f(s)$, então $y[i] = f(s[i])$
        \pause

        \item Em termos mais simples, $m_f$ mapeia cada caractere de $s$ de acordo com a função $f$
        \pause

        \item Por exemplo, se $A$ é formado pelas letras alfabéticas maiúsculas e minúsculas e 
            $f$ é a função \code{cpp}{toupper()}, o mapeamento $m_f$ tornaria maiúsculas todas as 
            letras de uma string $s$ dada
        \pause

        \item A implementação do mapeamento pode ser apenas conceitual, usando uma função padrão
            do C/C++, ou utilizar a função \code{c}{transform()} da STL da linguagem C++
    \end{itemize}

\end{frame}

\begin{frame}[fragile]{Exemplo de implementação de um mapa usando funções}
    \inputsnippet{cpp}{1}{19}{codes/map_function.cpp}
\end{frame}

\begin{frame}[fragile]{Implementação da cifra de César usando a função \texttt{transform()}}
    \inputsnippet{cpp}{1}{21}{codes/transform.cpp}
\end{frame}

\begin{frame}[fragile]{Filtros}

    \begin{itemize}
        \item Um filtro $f_p : S_N \to S_M$ consiste em uma função que gera uma string de tamanho 
            $M \leq N$, a partir de uma string de tamanho $N$, através da seleção dos $M$ 
            caracteres cujo predicado $p: A \to Bool$ retorna verdadeiro
        \pause

        \item Assim como os mapas, os filtros podem ser implementados de forma apenas
            conceitual, usando elementos da própria linguagem, como laços e condicionais
        \pause

        \item Outra maneira de implementar o mesmo código é utilizar a função \code{cpp}{copy_if()}
            da STL, que tem sintaxe semelhante a da função \code{c}{transform()}
        \pause

        \item A função \code{c}{remove_copy_if()} tem comportamento análogo, mas copia os 
            caracteres que negarem o predicado

    \end{itemize}

\end{frame}

\begin{frame}[fragile]{Implementação de um filtro usando elementos de C++}
    \inputcode{cpp}{codes/filtro.cpp}
\end{frame}
\begin{frame}[fragile]{Implementação de um filtro usando a função \code{c}{copy_if()}}
    \inputcode{cpp}{codes/copy_if.cpp}
\end{frame}

\begin{frame}[fragile]{Reduções}

    \begin{itemize}
        \item Uma redução $r_b : S_N \to T$ gera um elemento do tipo $T$ através da aplicação 
            sucessiva do operador binário $b: T\times \Sigma \to T$, da esquerda para a direita, em cada elemento de $s$, 
            tendo como operando esquerdo inicial um valor definido previamente
        \pause

        \item Se $b(t_i, s_j) = t_j$ e $t_0$ é o valor inicial para o operando esquerdo, 
            então a redução se comporta da seguinte maneira na sequência $s = \lbrace s_1, s_2,
                \ldots, s_N\rbrace$:
        \begin{align*}
            r_b(s) &= \lbrace b(t_0, s_1), s_2, \ldots, s_N\rbrace \\ 
            &= \lbrace b(t_1, s_2), \ldots, s_N\rbrace \\
            &= \lbrace b(t_2, s_3), \ldots, s_N\rbrace \\
            &= \ldots \\
            &= \lbrace b(t_{N-1}, s_N) \rbrace = t_N
        \end{align*}

    \end{itemize}

\end{frame}

\begin{frame}[fragile]{Reduções}

    \begin{itemize}
        \item Reduções podem ser implementadas de maneira natural usando um laço e um 
            acumulador
        \pause

        \item Uma alternativa é utilizar a função \code{c}{accumulate()} da STL, que
            abstrai o conceito de redução, parametrizando o tipo de retorno e o valor
            inicial do operando esquerdo
        \pause

        \item No padrão C++17 foi introduzida a função \code{c}{reduce()}
        \pause

        \item O comportamento é semelhante ao da função \code{c}{accumulate()}, mas ela
            não impõem uma ordem na aplicação do operador binário
        \pause

        \item Este relaxamento permite otimizações por parte do compilador e a execução
            em paralelo
    \end{itemize}

\end{frame}

\begin{frame}[fragile]{Implementação de uma redução usando laço e acumulador}
    \inputcode{cpp}{codes/reduce_loop.cpp}
\end{frame}

\begin{frame}[fragile]{Implementação de uma redução usando a função \code{c}{reduce()}}
    \inputcode{cpp}{codes/reduce.cpp}
\end{frame}

\begin{frame}[fragile]{Tabelas de substituição}

    \begin{itemize}
        \item Uma tabela de substituição é uma aplicação prática de um mapeamento
        \pause

        \item Ela é definida por uma função $f: \Sigma \to \Sigma$ que mapeia cada caractere do alfabeto 
            $\Sigma$ em outro caractere do mesmo alfabeto
        \pause

        \item Sistemas criptográficos mais simples utilizam tabelas de substituição, como a Cifra 
            de César e a criptografia baseada em ou exclusivo (\textit{xor})
        \pause

        \item Se $|A|$ não é muito grande, a função $f$ pode ser implementada em um \textit{array}
            estático, permitindo a consulta da substituição com complexidade $O(1)$
        \pause

        \item Caso contrário, ou se o alfabeto não é contíguo, deve ser usado um dicionário para 
            armazenar tais substituições
            
    \end{itemize}

\end{frame}

\begin{frame}[fragile]{Exemplo de uso de tabela de substituição}
    \inputsnippet{cpp}{1}{18}{codes/table.cpp}
\end{frame}

\begin{frame}[fragile]{Exemplo de uso de tabela de substituição}
    \inputsnippet{cpp}{20}{36}{codes/table.cpp}
\end{frame}

\begin{frame}[fragile]{Exemplo de uso de tabela de substituição}
    \inputsnippet{cpp}{38}{62}{codes/table.cpp}
\end{frame}
