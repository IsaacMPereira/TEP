\section{Algoritmo de Knuth-Morris-Pratt}


\begin{frame}[fragile]{Diferenças entre os algoritmos MP e KMP}

    \begin{itemize}
        \item O algoritmo de Morris-Pratt tem como invariante
        \[
            inv(i, j) = (P[1..j] = S[i + 1, i + j + 1]),
        \]
        o qual permite os saltos seguros e que resulta na complexidade $O(n + m)$ no pior caso

        \item Contudo, este invariante pode ser melhorada, ao se incorporar a propriedade da 
        diferença, isto é,
        \[
            inv2(i, j) = (P[1..j] = S[i + 1, i + j + 1])\ \mathbf{and}\ (P[j + 1] \neq S[i + j + 1])
        \]

        \item Para entender o porquê da melhora, considere o seguinte exemplo: seja 
            $P$ = \code{cpp}{"abcabc"} e $S$ = \code{cpp}{"abcabdabc"}

        \item Os 5 primeiros caracteres de ambos coincidem e a diferença ocorre no sexto 
            caractere: $P[6]$ = \code{cpp}{'c'} $\neq S[6]$ = \code{cpp}{'d'}

    \end{itemize}

\end{frame}

\begin{frame}[fragile]{Diferenças entre os algoritmos MP e KMP}

    \begin{itemize}
        \item Como $border(P[1..5])$ = \code{cpp}{"ab"}, cujo tamanho é igual a 2, a próxima 
            comparação seria entre $P[3]$ e $S[6]$

        \item Contudo, esta comparação é idêntica a anterior, pois a borda \code{cpp}{"ab"} não é 
            própria, isto é, o próximo caractere (\code{cpp}{'c'}) gera uma nova borda 
            \code{cpp}{"abc"}

        \item A contribuição de Knuth para o algoritmo de Morris-Pratt é essa: incorporar a 
            propriedade da diferença e definir as bordas estritas, nas quais o próximo caractere 
            não gera uma nova borda, evitando comparações já realizadas
    \end{itemize}

\end{frame}


\begin{frame}[fragile]{Bordas estritas}

    \begin{itemize}
        \item Uma borda estrita de $S[1..j]$ ($sborder(S[1..j]))$) é a maior borda própria 
            $b = [1..k]$ de $S$ tal que $S[j + 1] \neq S[k + 1]$

        \item Assim, os coeficientes $b_j$ podem ser redefinidos como
        \[
            b_j = |sborder(S[1..j])|,
        \]
        ou $b_j = -1$, caso não exista tal borda

        \item Observe que a borda $b$ pode ser uma string vazia

        \item Esta modificação melhora o tempo de execução de algoritmo, embora sua 
            complexidade assintótica permaneça a mesma: $O(n + m)$

        \item A implementação é quase idêntica a do algoritmo de Morris-Pratt, residindo a 
            diferença na substituição de $border(P)$ por $sborder(P)$

    \end{itemize}

\end{frame}

\begin{frame}[fragile]{Exemplo de bordas estritas e de saltos seguros}

    \begin{center}
    \begin{tabularx}{0.9\textwidth}{cXcc}
        \toprule
        $j$ & $P[1..j]$ & $sborder(P[1..j])$ & $shift(P, j)$ \\
        \midrule
            0&\textcolor{red!70}{\verb|""|}&-1&1\\
            1&\textcolor{red!70}{\verb|"a"|}&0&1\\
            2&\textcolor{red!70}{\verb|"ab"|}&-1&3\\
            3&\textcolor{red!70}{\verb|"aba"|}&0&3\\
            4&\textcolor{red!70}{\verb|"abab"|}&2&2\\
            5&\textcolor{red!70}{\verb|"ababb"|}&-1&6\\
            6&\textcolor{red!70}{\verb|"ababba"|}&0&6\\
            7&\textcolor{red!70}{\verb|"ababbab"|}&-1&8\\
            8&\textcolor{red!70}{\verb|"ababbaba"|}&0&8\\
            9&\textcolor{red!70}{\verb|"ababbabab"|}&4&5\\
            10&\textcolor{red!70}{\verb|"ababbababa"|}&0&10\\
            11&\textcolor{red!70}{\verb|"ababbababab"|}&4&7\\
        \bottomrule
    \end{tabularx}
    \end{center}

\end{frame}

\begin{frame}[fragile]{Implementação do algoritmo KMP em C++}
    \inputsnippet{cpp}{1}{20}{codes/kmp.cpp}
\end{frame}

\begin{frame}[fragile]{Implementação do algoritmo KMP em C++}
    \inputsnippet{cpp}{22}{41}{codes/kmp.cpp}
\end{frame}
