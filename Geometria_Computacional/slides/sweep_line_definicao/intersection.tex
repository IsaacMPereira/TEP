\section{Interseção de intervalos}

\begin{frame}[fragile]{Interseção de intervalos}

    \begin{itemize}
        \item Um problema que pode ser resolvido usando o paradigma \textit{sweep line} é o de
            contabilizar, dentre um conjunto de intervalos $I_i = [a_i, b_i)$, o tamanho do maior 
            subconjunto $S$ destes intervalos tal que $I_j \cap I_k \neq \emptyset$ para todo
            par $I_j, I_k\in S$ 
        \pause

        \item Uma aplicação prática deste problema seria: cada intervalo representa o início e o
            fim de um espetáculo que acontecerá em determinado dia. Qual é o número máximo de
            espetáculos que acontecerão simultaneamete?
        \pause

        \item A solução é criar, para cada intervalo, dois eventos: um evento de ínicio do 
            espetáculo $(a_i, 1)$ e um evento de final $(b_i, 0)$
        \pause

        \item Uma vez ordenados estes eventos em ordem lexicográfica (primeiro por coordenada $x$,
            depois por coordenada $y$), basta processar todos eles, um por vez

    \end{itemize}

\end{frame}

\begin{frame}[fragile]{Interseção de intervalos}

    \begin{itemize}
        \item Um evento de início incrementa o número de eventos em andamento, o evento de fim
            decrementa
        \pause

        \item Com a representação de eventos escolhida, os intervalos $[a, b)$ e $[b, c)$ não tem
            interseção
        \pause

        \item A complexidade deste algoritmo é $O(N\log N)$, por conta da ordenação, pois
            cada ponto será processado uma única vez
        \pause

        \item A representação dos eventos pode ser modificada, de tal modo que é possível 
            identificar quais são os intervalos simultâneos
        \pause

        \item Basta fazer $(a_i, i)$ e $(b_i, -i)$ e manter os índices dos intervalos em
            exibição em um conjunto, removendo-os a cada evento de encerramento
        \pause

        \item Veja que, nesta represenção, os intervalos devem ser numerados a partir de 1,
            pois o zero geraria ambiguidade
    \end{itemize}

\end{frame}

\input{inter}

\begin{frame}[fragile]{Implementação da interseção de intervalos}
    \inputsnippet{cpp}{1}{18}{codes/intersection.cpp}
\end{frame}

\begin{frame}[fragile]{Implementação da interseção de intervalos}
    \inputsnippet{cpp}{20}{36}{codes/intersection.cpp}
\end{frame}
